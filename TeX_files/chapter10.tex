\chapter{Analysis of Variance}


\begin{flushright}
	\textit{``What do you mean `Its a method used to compare mean values'?"} \\
\end{flushright}

\section{Introduction}

Consider the problem of testing whether alternative approaches to solving a problem are equivalent. A simple null hypothesis is to ask if the average benefit of each problem-solving approach is the same. \\
	
A standard procedure to test this would be to randomly divide a population into many subgroups, each subjected to a different solution. The null hypothesis, under the assumption that the response of each individual tested has a variance independent of the solution itself, is simply the subgroup-means being equal.\\

\section{Overview of the procedure}

Consider a set of $ m $ populations from which samples of size $ n $ are drawn. Proving the equality of the means of these populations, which only depend on one parameter, namely the population the sample was from, is called a \textit{one-way} analysis of variance. \\

This can also be extended to the more general case of the $ m $ different samples not being of equal size. \\

The more complex case of the variance of each variable depending on two factors, the population index as well as the sample index, can be visually represented as a two-dimensional array with row and column indices both determining the variance. This is called the \textit{two-way} analysis of variance problem. \\

The above problem assumes no interaction between the two factors affecting the mean of the variable. Removing this assumption makes for a significantly more complicated problem, with possible non-linear interactions among the factors affecting the mean.\\

For the rest of this chapter, a general procedure can be outlined as follows,\\

\begin{itemize}
	\item Assume samples are drawn from a normal population with unknown (but common) variance $ \sigma^2 $
	\item Find a valid estimator of $ \sigma^2 $, which is true regardless of the specific null hypothesis being true.
	\item Fina another estimator of $ \sigma^2 $ which is true only when $ H_0 $ holds.
	\item Design a test that rejects $ H_0 $ when the second estimator is sufficiently larger than the first (since it tends to always be larger). 
\end{itemize}


Consider a set of $ N $ independent normal RVs $ \{X_i\} $ with possibly different means but a common unknown variance. 

\begin{align}
	\frac{X_i - \mu_i}{\sigma} &\sim Z_i \nonumber \\
	%
	\sum\limits_{i = 1}^{N} \frac{(X_i - \mu_i)^2}{\sigma^2} &\sim \chi^2_{N} \nonumber \\
\end{align}\\

If each $ \mu_i  = \mathbb{E}[X_i]$ can be estimated using a linear combination of $ k $ parameters, then the resulting estimate of the means $ \{\widehat{\mu_i}\} $, can be substituted into the above expression to make a $ \chi^2 $ distribution with $ (N-k) $ DOF.\\

\section{One-way ANOVA}

Consider a set of $ m $ samples, each of size $ n $ (\textit{balanced samples}). The members of each sample are independent normal RVs and a hypothesis testing the equality of their means, \\

\begin{align}
	X_{ij} &\sim \mathcal{N}(\mu_i, \sigma^2) \qquad i = 1,\dots,m \qquad j = 1,\dots,n \\
	%
	H_0 &: \mu_1 = \mu_2 = \dots = \mu_m \qquad \text{vs} \qquad H_1 : \text{not all means are equal}
\end{align}\\

Using the set of sample means $ \{X_{i\bullet}\} $ of the $ n $ populations as the estimators for $ \{\mu_i\} $, \\

\begin{align}
	\sum\limits_{i = 1}^{m} \sum\limits_{j = 1}^{n} \frac{(X_{ij} - \mathbb{E}[X_{ij}])^2}{\sigma^2} &\sim \chi^2_{nm} \\
	%
	\sum\limits_{i = 1}^{m} \sum\limits_{j = 1}^{n} \frac{(X_{ij} - X_{i\bullet})^2}{\sigma^2} &\sim \chi^2_{nm - m} 
\end{align}\\

Defining the \textit{within-samples sum of squares} ($ SS_W $) as 

\begin{align}
	SS_W &= \sum_{i} \sum_{j} (X_{ij} - X_{i\bullet})^2 \\
	%
	\sum_{i} \sum_{j} \frac{SS_W}{\sigma^2} &\sim \chi^2_{nm-m} \\
	%
	\mathbb{E}\left[\frac{SS_W}{\sigma^2}\right] &= nm - m \\
	%
	\sigma^2 &= \frac{SS_W}{nm-m}
\end{align}\\

The above estimator of $ \sigma^2 $ is free of any assumptions about the $ H_0 $. Next, under the assumption that $ H_0 $ holds, and thus the set of population means $ \{X_{i\bullet}\} $ are all normal RVs with common mean $ \mu $ and common variance $ \sigma^2 / n $, \\

\begin{align}
	n\ \sum\limits_{i = 1}^{m} \frac{(X_{i\bullet} - \mu)^2}{\sigma^2} &\sim \chi^2_{m} \\
	%
	X_{\bullet \bullet} &= \frac{\sum_{i} \sum_{j} X_{ij}}{nm} = \frac{\sum_{i} X_{i\bullet}}{m} \\
\end{align}\\

Since the above mean of all variables $ X_{\bullet \bullet} $ is an estimator of the common mean $ \mu $ which requires only $ 1 $ parameter to calculate, the \textit{between-samples sum of squares} ($ SS_b $) can be used to simplify the above expression,\\

\begin{align}
	SS_b &= n\ \sum\limits_{i = 1}^{m} (X_{i\bullet} - X_{\bullet\bullet})^2 \\
	%
	\frac{SS_b}{\sigma^2} &\sim \chi^2_{m-1} \nonumber \\
	%
	\frac{SS_b}{(m-1)} &= \sigma^2
\end{align}\\

Since the conditional estimator of $ \sigma^2 $ tends to be larger and these two estimators are independent when $ H_0 $ holds, the ratio of these two $ \chi^2 $ RVs to define an f-RV, the hypothesis test can be formulated as,\\

\begin{align}
	\frac{SS_b}{SS_W}\ \frac{(nm-m)}{(m-1)} &\sim F_{m-1, nm-m} \nonumber \\
	%
	H_0 : \text{all means are equal} \qquad &\text{vs.} \qquad H_1 : \text{not all means are equal} \nonumber \\
	%
	\text{reject $ H_0 $ if } \qquad & \frac{SS_b}{SS_W}\ \frac{(nm-m)}{(m-1)} > F_{\alpha, m-1, nm-m}  \nonumber \\
	%
	\text{accept $ H_0 $} \qquad & \text{otherwise}
\end{align}\\

A computational identity relating the two terms $ SS_b $ and $ SS_W $ is as follows,\\

\begin{align}
	\sum\limits_{i = 1}^{m} \sum\limits_{j = 1}^{n} X^2_{ij} &= nm\ X^2_{\bullet \bullet} + SS_b + SS_W
\end{align} \\

A proof of the relation between the two estimates of $ \sigma^2 $, uses the following variable transformation, \\

\begin{align}
	\mathbb{E}\left[\frac{SS_b}{(m-1)}\right] &\geq \sigma^2 \qquad \text{equality only if $H_0$ holds} \\
	%
	Y_i &= X_{i\bullet} -\mu_i + \mu_{\bullet} \nonumber \\
	%
	\mathbb{E}\left[\frac{\sum_{i} (X_{i\bullet} - X_{\bullet\bullet})^2}{(m-1)}\right] &= \frac{\sigma^2}{n} + \sum_{i} \frac{(\mu_i - \mu_{\bullet})^2}{(m-1)}
\end{align}\\

\textit{Multiple comparisons of sample means} : When the null hypothesis above is rejected, a measure of how the different sample means are related is the T-method. This gives a confidence interval for the difference between all possible pairs $ \mu_i - \mu_j $. 

\begin{align}
	\mu_i - \mu_j &\in X_{i\bullet} - X_{j\bullet} \pm W \\
	%
	\forall \ i\neq j &\qquad \text{with probability $ 1-\alpha $} \nonumber \\
	%
	W &= \frac{1}{\sqrt{n}} \ q(m, nm-m, \alpha)\ \sqrt{\frac{SS_W}{(nm-m)}}
\end{align}\\

Here, the studentized range distribution $ q $ has pre-calculated CDF tables. Further information in article - \textit{Tukey's range test}.\\

\textit{Unequal sample sizes (unbalanced)} : Since the assumption of sample sizes all being equal to $ n $ is no longer valid, it is replaced with sample sizes $ \{n_1, n_2, \dots, n_m\} $ for each of the $ m $ samples.\\

For convenience, define $ N = \sum_i n_i $.\\

\begin{align}
	\sum\limits_{i = 1}^{m} \sum\limits_{j = 1}^{n_i} \frac{(X_{ij} - \mu_i)^2}{\sigma^2} &\sim \chi^2_{N} \\
	%
	\sum\limits_{i = 1}^{m} \sum\limits_{j = 1}^{n_i} \frac{(X_{ij} - X_{i\bullet})^2}{\sigma^2} &\sim \chi^2_{N - m} \\
\end{align}\\

Once again, defining $ SS_W $ and simplifying yields an unbiased estimator of $ \sigma^2 $ not dependent on $ H_0 $.\\

\begin{align}
	SS_W &= \sum\limits_{i = 1}^{m} \sum\limits_{j = 1}^{n_i} (X_{ij} - X_{i\bullet})^2 \\
	%
	\frac{SS_W}{N-m} &= \sigma^2
\end{align}\\

Next, under the assumption that $ H_0 $ holds, and the means are equal to a common value $ \mu $.\\

\begin{align}
	\mathbb{E}[X_{i\bullet}] &= \mu & \mathrm{Var}(X_{i\bullet}) &= \frac{\sigma^2}{n_i} \nonumber \\
	%
	\sum\limits_{i = 1}^{m} \frac{(X_{i\bullet} - \mu)^2}{\sigma^2 / n_i} &\sim \chi^2_{m} & \sum\limits_{i = 1}^{m} \frac{(X_{i\bullet} - X_{\bullet\bullet})^2}{\sigma^2 / n_i} &\sim \chi^2_{m-1} \\
	%
	SS_b &= \sum\limits_{i = 1}^{m} n_i\ (X_{i\bullet} - X_{\bullet\bullet})^2
\end{align}\\

When $ H_0 $ is true, both the estimators of $ \sigma^2 $ are unbiased and independent. This enables the construction of a level $ \alpha $ hypothesis test\\

\begin{align}
	\frac{SS_b}{SS_W}\ \frac{(N-m)}{(m-1)} &\sim F_{m-1, N-m} \nonumber \\
	%
	H_0 : \text{all means are equal} \qquad &\text{vs.} \qquad H_1 : \text{not all means are equal} \nonumber \\
	%
	\text{reject $ H_0 $ if } \qquad & \frac{SS_b}{SS_W}\ \frac{(N-m)}{(m-1)} > F_{\alpha, m-1, N-m}  \nonumber \\
	%
	\text{accept $ H_0 $} \qquad & \text{otherwise}
\end{align}\\

Balanced samples are more robust to violations of the assumption of equal variances of the different populations and thus preferred over unbalanced samples.\\

\section{Two-factor ANOVA}

Consider the more complex where each data value in the data-set is affected by two factors, with ($ m, n $) \textit{levels} respectively. The dataset can be arranged into a two-dimensional array along these two indices.\\

\begin{align}
	\begin{bmatrix}
		X_{11} & X_{12} & X_{13} & \dots & X_{1n} \\
		X_{21} & X_{22} & X_{23} & \dots & X_{2n} \\
		\vdots & \vdots & \vdots & \ddots & \vdots \\
		X_{i1} & X_{i2} & X_{i3} & \dots & X_{in} \\
		\vdots & \vdots & \vdots & \ddots & \vdots \\
		X_{m1} & X_{m2} & X_{m3} & \dots & X_{mn} \\
	\end{bmatrix}
\end{align}\\

By convention the factors are called the \textit{row-factor} and \textit{column-factor} respectively. The mean value of a data point depends additively on both its row and column index.\\

Let $ \mu_{ij} = \mathbb{E}[X_{ij}] $, where $ i \in \{1, \dots, m\} $ and $ j \in \{1, \dots, n\} $.\\

\begin{align}
	\mu_{ij} &= a_i + b_j & \mu = \mu_{\bullet\bullet} &= \sum\limits_{i = 1}^{m} \sum\limits_{j = 1}^{n} \frac{\mu_{ij}}{nm} \\
	%
	\mu_{\bullet j} &= a_{\bullet} + b_j & \mu_{i \bullet} &= a_{i} + b_{\bullet} \\
	%
	\alpha_i &= a_i - a_{\bullet} & \beta_j &= b_j - b_{\bullet}
\end{align}\\

The \textit{grand mean} is defined as the mean of all terms $ \mu \coloneqq \mu_{\bullet \bullet} $, and the terms $ \alpha_i, \beta_j $ are the row and column \textit{deviations} from the grand mean.\\

The model can now be rewritten as\\

\begin{align}
	\mu_{ij} = \mathbb{E}[X_{ij}] &= \mu + \alpha_i + \beta_j \\
	%
	\sum_{i} \alpha_i &= \sum_{j} \beta_j = 0 
\end{align}\\

Using the row-average, column-average and overall-average of the data points, $ X_{i\bullet},\ X_{\bullet j},\ X_{\bullet \bullet} $ to calculate the estimators of $ \alpha_i, \beta_j, \mu $ respectively,\\

\begin{align}
	\mathbb{E}[X_{i\bullet}] &= \mu + \alpha_i & \mathbb{E}[X_{\bullet j}] &= \mu + \beta_j \\
	%
	\mathbb{E}[X_{i\bullet} - X_{\bullet \bullet}] &= \alpha_i & \mathbb{E}[X_{\bullet j} - X_{\bullet \bullet}] &=  \beta_j 
\end{align}\\

This leads to the following unbiased estimators of the model parameters.

\begin{align}
	\widehat{\mu} &= X_{\bullet \bullet} \nonumber \\
	%
	\widehat{\alpha_i} &= X_{i\bullet} - X_{\bullet \bullet} \nonumber \\
	%
	\widehat{\beta_j} &= X_{\bullet j} - X_{\bullet \bullet}
\end{align}\\

\section{Two-factor ANOVA : Hypothesis testing} 

The most common hypothesis to be tested is the absence of any effect due to the row-factors or the column-factors. \\

\begin{align}
	H_0 : \text{all $ \alpha_i $ are zero} \qquad \text{vs} \qquad H_1 : \text{not all $ \alpha_i $ are zero} \nonumber \\
	H_0 : \text{all $ \beta_j $ are zero} \qquad \text{vs} \qquad H_1 : \text{not all $ \beta_j $ are zero} \nonumber
\end{align}\\

Using the usual ANOVA procedure of comparing two different estimators of the variance, \\

\begin{align}
	\sum\limits_{i = 1}^{m} \sum\limits_{j = 1}^{n} \frac{(X_{ij} - \mathbb{E}[X_{ij}])^2}{\sigma^2} &\sim \chi^2_{mn} \nonumber \\
	%
	\sum\limits_{i = 1}^{m} \sum\limits_{j = 1}^{n} \frac{(X_{ij} - \widehat{\mu} - \widehat{\alpha_i} - \widehat{\beta_j})^2}{\sigma^2} &= \sum\limits_{i = 1}^{m} \sum\limits_{j = 1}^{n} \frac{(X_{ij} + X_{\bullet \bullet} - X_{i \bullet} - X_{\bullet j})^2}{\sigma^2}
\end{align}\\

The number of parameters required to estimate the above 3 parameters is $ 1 + (m-1) + (n-1) = (m+n-1)$. This leaves a $ \chi^2 $ RV with $ mn - (m+n-1) = (m-1)(n-1) $ DOF. Using the definiton of the \textit{error sum of squares},\\

\begin{align}
	SS_e &=	\sum\limits_{i = 1}^{m} \sum\limits_{j = 1}^{n} (X_{ij} - X_{\bullet \bullet} - X_{i \bullet} - X_{\bullet j})^2\\
	%
	\frac{SS_e}{(m-1)(n-1)} &= \sigma^2
\end{align}\\

To find the $ H_0 $ conditioned estimator of $ \sigma^2 $, first consider the case where all of the row-factors are zero.

\begin{align}
	\mathbb{E}[X_{i\bullet}] &= \mu + \alpha_i = \mu & \mathrm{Var}(X_{i\bullet}) &= \frac{\sigma^2}{n} \nonumber \\
	%
	\text{if $ H_0 $ is true, } && n\sum\limits_{i = 1}^{m} \frac{(X_{i\bullet}  - \mu)^2}{\sigma^2} &\sim \chi^2_{m} \\
	%
	n\sum\limits_{i = 1}^{m} \frac{(X_{i\bullet}  - X_{\bullet \bullet})^2}{\sigma^2} &\sim \chi^2_{m-1}
\end{align}\\

Defining the \textit{row sum of squares} and analogously the \textit{column sum of squares}, \\

\begin{align}
	SS_r &= n\sum\limits_{i = 1}^{m} (X_{i\bullet}  - X_{\bullet \bullet})^2 & SS_c &= m\sum\limits_{j = 1}^{n} (X_{\bullet i}  - X_{\bullet \bullet})^2 
\end{align}\\

Using the above definitions, a second estimator for $ \sigma^2 $ conditioned on $ H_0 $ being true is,\\

\begin{align}
	\frac{SS_r}{(m-1)} &= \sigma^2 & \frac{SS_r}{SS_e}\ \frac{(m-1)(n-1)}{(m-1)} &\sim F_{m-1, (m-1) (n-1)}
\end{align}\\

The hypothesis test can now be constructed using this F-RV as follows,\\

\begin{align}
	H_0 : \text{all row-factors are zero} \qquad &\text{vs.} \qquad H_1 : \text{not all row-factors are zero} \nonumber \\
	%
	\text{reject $ H_0 $ if } \qquad & \frac{SS_r}{SS_e}\ \frac{(m-1)(n-1)}{(m-1)} > F_{\alpha, m-1, (m-1) (n-1)} \nonumber \\
	%
	\text{accept $ H_0 $} \qquad & \text{otherwise}
\end{align}\\

\section{Two-way ANOVA with interaction} 

Extending the model in the previous section to allow the possibility of some interaction between the row and column factors governing a particular data-point, \\

\begin{align}
	\mu_{ij} &= \mathbb{E}[X_{ij}] & \mu_{ij} &= \mu + \alpha_i + \beta_j + \gamma_{ij} \\
	%
	\mu &= \mu_{\bullet \bullet} & \gamma_{ij} &= \mu_{ij} - \mu_{i\bullet} - \mu_{\bullet j} + \mu_{\bullet \bullet} \\
	%
	\alpha_i &= \mu_{i\bullet} - \mu_{\bullet \bullet} & \beta_j &= \mu_{\bullet j} - \mu_{\bullet \bullet} \\
	%
	\sum\limits_{i = 1}^{m} \gamma_{ij} &= 0 & \sum\limits_{j = 1}^{n} \gamma_{ij} &= 0
\end{align}\\

In addition, the usual constraints on the set of $ \{\alpha_i\} $ and $ \{\beta_j\} $ summing to zero still apply. In this extended mode, the term $ \gamma_{ij} $ is called the \textit{interaction of row i and column j}. It measures the difference between the mean of $ X_{ij} $ and the three terms indicating the grand mean, row effect and column effect.\\

For mathematical reasons, it is no longer sufficient to have just one observation for a given row and column index. Assume there are a set of $ l $ such observations corresponding to every possible pair of $ \{i, j\} $. Since each data point $ X_{ijk} $ is still a normal RV with common unknown variance $ \sigma^2 $,\\

\begin{align}
	X_{ijk} &\sim \mathcal{N}(\mu + \alpha_i + \beta_j + \gamma_{ij},\ \sigma^2) & k &\in \{1, \dots, l\} \\
	%
	i &\in \{1, \dots, m\} & j &\in \{1, \dots n\} \nonumber 
\end{align}\\

To find the estimators for the three sets of parameters defined above, use,\\

\begin{align}
	\mathbb{E}[X_{ij\bullet}] &= \mu_{ij} & \mathbb{E}[X_{\bullet \bullet \bullet}] &= \mu \\
	%
	\mathbb{E}[X_{i\bullet \bullet}] &= \mu + \alpha_i & \mathbb{E}[X_{\bullet j \bullet}] &= \mu + \beta_j
\end{align}\\

Using the fact that substituting the parameters above with their unbiased estimators will lead to a reduction in DOF of the $ \chi^2 $ RV,\\

\begin{align}
	\widehat{\mu} &= X_{\bullet \bullet \bullet} \nonumber \\
	%
	\widehat{\alpha_i} &= X_{i\bullet \bullet} - X_{\bullet \bullet \bullet} \nonumber \\
	%
	\widehat{\beta_j} &= X_{\bullet j \bullet} - X_{\bullet \bullet \bullet} \nonumber \\
	%
	\widehat{\gamma_{ij}} &= X_{ij\bullet} - X_{i\bullet \bullet} - X_{\bullet j \bullet} + X_{\bullet \bullet \bullet} 
\end{align}\\

For hypothesis tests asking whether one of the three sets of parameters are all zeros, arrange $ \chi^2 $RVs as follows,\\

\begin{align}
	\sum\limits_{i = 1}^{m} \sum\limits_{j = 1}^{n} \sum\limits_{k = 1}^{l}\frac{(X_{ijk} - \mathbb{E}[X_{ijk}])^2}{\sigma^2} &\sim \chi^2_{mnl} \nonumber \\
	%
	\sum\limits_{i = 1}^{m} \sum\limits_{j = 1}^{n} \sum\limits_{k = 1}^{l} \frac{(X_{ij} - \widehat{\mu} - \widehat{\alpha_i} - \widehat{\beta_j} - \widehat{\gamma_{ij}})^2}{\sigma^2} &\sim \chi^2_{mnl - mn}
\end{align}\\

The number of parameters needed to estimate $ \widehat{\mu},\ \widehat{\alpha_i},\ \widehat{\beta_j},\  \widehat{\gamma_{ij}} $ are $ 1,\ (m-1),\ (n-1),\ (m-1)(n-1) $ respectively. This means that the loss in DOF is $ mn $.

Now, defining the \textit{error sum of squares} $ SS_e $ as\\

\begin{align}
	SS_e &= \sum\limits_{i = 1}^{m} \sum\limits_{j = 1}^{n} \sum\limits_{k = 1}^{l} (X_{ijk} - X_{ij\bullet})^2 \\
	%
	\frac{SS_e}{(mnl  - mn)} &= \sigma^2
\end{align}\\

Having found the true estimator of the variance, now consider the estimator conditioned on the null hypothesis that there are no interaction terms. Now, each of the variables $ X_{ij\bullet} $ is averaged over $ l $ data points, and thus\\

\begin{align}
	H_0^{\textit{int}} : \gamma_{ij} &= 0 & \forall\ &i,j \nonumber \\
	%
	\mathbb{E}[X_{ij\bullet}] &= \mu + \alpha_i + \beta_j & \mathrm{Var}(X_{ij\bullet}) &= \frac{\sigma^2}{l} \\
	%
	\sum\limits_{i = 1}^{m} \sum\limits_{j = 1}^{n} \frac{(X_{ij\bullet} - \mu - \alpha_i - \beta_j)^2}{\sigma^2 / l} &\sim \chi^2_{mn}
\end{align} \\

From an earlier section, replacing the parameters above with their estimators involes an $ m+n-1 $ loss in DOF. Defining the \textit{non-interaction sum of squares} ($ SS_{int} $) \\

\begin{align}
	SS_{int} &= \sum\limits_{i = 1}^{m} \sum\limits_{j = 1}^{n}\ l\ (X_{ij\bullet} - X_{i\bullet \bullet} - X_{\bullet j \bullet} + X_{\bullet \bullet \bullet})^2 \\
	%
	\frac{SS_{int}}{(n-1)(m-1)} &= \sigma^2 \qquad \text{if $ H_0^{int} $ holds}
\end{align} \\

Using the two estimators of $ \sigma^2 $ above, an F-RV can be created to test the null hypothesis at significance level $ a $ as follows,\\

\begin{align}
	H_0^{int} : \text{all $ \gamma{ij} $ are zero} \qquad &\text{vs.} \qquad H_1^{int} \nonumber \\
	%
	\text{reject $ H_0 $ if } \qquad & \frac{SS_{int}}{SS_e}\ \frac{(mnl - mn)}{(n-1)(m-1)} > F_{a,(m-1)(n-1), mnl-mn} \nonumber \\
	%
	\text{accept $ H_0 $} \qquad & \text{otherwise}
\end{align}\\

Similar hypothesis tests for row-factors or column-factors being all zero, involve the \textit{row sum of squares} and \textit{column sum of squares} respectively. For example the null hypothesis $ H_0^r : \text{all $ \alpha_i $ are zero} $ can be tested using\\

\begin{align}
	SS_r &= \sum\limits_{i = 1}^{m} \ nl\ (X_{i\bullet \bullet} - X_{\bullet \bullet \bullet})^2 \\
	%
	\frac{SS_{r}}{SS_e}\ \frac{(mnl - mn)}{(m-1)} &\sim F_{m-1, mnl-mn} 
\end{align}\\

An analogous expression can be written for the null hypothesis $ H_0^c : \text{all $ \beta_j $ are zero} $.\\

Note that the procedure in the later sections is a straightforward increase in complexity from the same procedure in earlier sections which was outlined in the introduction as the general ANOVA approach.